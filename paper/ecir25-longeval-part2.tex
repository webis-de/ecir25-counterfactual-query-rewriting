\section{Related Work}


\begin{table}
    \caption{Taxonomy reproduced and adapted from~\cite{keller:2024}.}
    \label{tab:CRUD}
    \begin{tabularx}{\linewidth}{l | X | X | X}
        \toprule
        {}       & \texttt{CREATE}          & \texttt{UPDATE}                 & \texttt{DELETE} \\ \midrule
        Document & New unseen documents     & Known URL but changed website   & URL not in sub-collection anymore              \\\midrule
        Topic    & New query                & Fixed typos, translation or changed accents marks                               & Query not in sub-collection anymore \\\midrule
        Qrel     & New relevance label      & Known document query pair but changed relevance & Relevance became unknown   \\\bottomrule
    \end{tabularx}
\end{table}

\paragraph{Temporal Information Retrieval}

Changes in evolving test collections are classified on a high level along theire main components, the documents, topics, and qrels, and the create, update, and delete operation of persistent storage by Keller et al. in~\cite{keller:2024}. Based on this classification schema different retrieval scenarios emerge that can be quantified on different levels and directly related to challenges IR systems face.


While the LongEval dataset contains changes in all components of all types, the investigated approaches are mainly concerned with changing documents that may also affect the relevance label.

% Table 2 refers to the scenario of D'TQ -> documents changes but relevance stayed the same


Things that sound related but are not yet checked/incorporated are appended to this sentence~\cite{li:2022}.

Integrate prior work into the topology by Keller et al.~\cite{keller:2024}. E.g., transfer of relevance labels to (near-)duplicate documents~\cite{froebe:2021a}, documents that change their content and therefore might be not relevant anymore~\cite{froebe:2022d}, etc (Harrie Oosterhuis likely has much on this).



\paragraph{Web Dynamics.} Web page changes every 90 days on average...

\paragraph{Caching in IR.}


\paragraph{Query Rewriting with Keyqueries.} {\color{red} Maik.}

