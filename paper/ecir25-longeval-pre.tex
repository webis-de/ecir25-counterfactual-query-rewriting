\title{Counterfactual Query Rewriting \\ for Historical Relevance Feedback}
%\title{Query Reformulation for Historical Relevance Data}
%\title{Query Reformulation for Recurring Search Queries}
%\title{Query Reformulation for Recurring Web Search Queries}
%\title{Rewriting Queries for Incorporating Historical Relevance Data}

%\titlerunning{Abbreviated paper title}

\author{First Author\inst{1} \and
Maik's Mic\inst{2,3} \and
Third Author\inst{3}}
%
\authorrunning{F. Author et al.}
% First names are abbreviated in the running head.
% If there are more than two authors, 'et al.' is used.
%
\institute{Princeton University, Princeton NJ 08544, USA \and
Springer Heidelberg, Tiergartenstr. 17, 69121 Heidelberg, Germany
\email{lncs@springer.com}}
%
\maketitle              % typeset the header of the contribution
%
\begin{abstract}
Search engines have seen many of the submitted queries before, which might provide valuable relevance feedback for recurring queries that did not change their intent. However, this relevance feedback can often not be applied directly, as documents that were previously relevant to a query might not exist anymore or might have substantially changed content. By counterfactually assuming that the previously relevant document still exists, we can formulate so-called keyqueries so that the previously relevant documents, if they would still exist, would be retrieved at the top positions of the ranking. Our evaluations in the LongEval scenario with varying time gaps between 1~month to 1~year and naturally and simulated removed/changed documents show that XYZ.

\keywords{Query Rewriting \and Longtitutal evaluation \and Another keyword.}
\end{abstract}
